% !TEX encoding = UTF-8
%Koma article
\documentclass[fontsize=12pt,paper=letter,twoside]{scrartcl}
\usepackage{float}
\usepackage{listings}
\usepackage{makecell}

%Standard Pre-amble
\usepackage[top=4cm,bottom=4cm,left=3cm,right=3cm,asymmetric]{geometry}
%\geometry{landscape}                % Activate for for rotated page geometry
%\usepackage[parfill]{parskip}    % Begin paragraphs with an empty line rather than an indent
\usepackage[table,xcdraw]{xcolor}
\usepackage{graphicx}

\usepackage{amsmath}
\usepackage{amssymb}
\usepackage{epstopdf}
\DeclareGraphicsRule{.tif}{png}{.png}{`convert #1 `dirname #1`/`basename #1 .tif`.png}
% Listings needs package courier
\usepackage{listings} % Needs 
\usepackage{courier}

\usepackage[framemethod=TikZ]{mdframed}
\usepackage{url}

\usepackage{sty/bsymb} %% Event-B symbols
\usepackage{sty/eventB} %% REQ and ENV
\usepackage{sty/calculation}

%Maths
\usepackage{amssymb,amsmath}
\def\Fl{\mathbb{F}}
\def\Rl{\mathbb{R}}
\def\Nl{\mathbb{N}}
\def\Bl{\mathbb{B}}
\def\St{\mathbb{S}}
\newcommand{\ovr}{\upharpoonright}
\newcommand{\var}[1]{\textit{#1}}
%Useful definitions
\newcommand{\mv}[1]{\textit{m\_#1}}
\newcommand{\cv}[1]{\textit{c\_#1}}
\newcommand{\degree}[1]{^{\circ}\mathrm{#1}}
%\newcommand{\comment}[1]{{\footnotesize \quad\texttt{--}\textrm{#1}}}
\newcommand{\im}[1]{i\texttt{-\!#1}}

\usepackage[headsepline]{scrpage2}
\pagestyle{scrheadings}
\ihead[]{\small EECS4312 Report1}
\ohead[]{\small \thepage}
\cfoot[]{}
\ofoot[]{}


%%%%PVS environment%%%%%%%%%%%%%%%%%%%
\lstnewenvironment{pvs}[1][]
    {\lstset{#1,captionpos=b,language=pvs,
    mathescape=true,
    basicstyle=\small\ttfamily,
    numbers=none,
    frame=single,
    % numberstyle=\tiny\color{gray},
    % backgroundcolor=\color{lightgray},
    firstnumber=auto
    }}
    {}
 %%%%%%%%%%%%%%%%%%%%%%%%%%%%%%%%
 
%%%%Verbatim environment%%%%%%%%%%%%%%%%%%%
\lstnewenvironment{code}[1][]
    {\lstset{#1,captionpos=b,
    mathescape=true,
    basicstyle=\small\ttfamily,
    numbers=none,
    frame=single,
    % numberstyle=\tiny\color{gray},
    % backgroundcolor=\color{lightgray},
    firstnumber=auto
    }}
    {}

% \newenvironment{boxed}[1]
%    {\begin{center}
%    #1\\[1ex]
%    \begin{tabular}{|p{0.9\textwidth}|}
%    \hline\\
%    }
%    { 
%    \\\\\hline
%    \end{tabular} 
%    \end{center}
%    }
 %%%%%%%%%%%%%%%%%%%%%%%%%%%%%%%%
 
 %Text in a box
\newenvironment{textbox}
    {\begin{center}
    \begin{tabular}{|p{0.9\textwidth}|}
    \hline\\
    }
    { 
    \\\\\hline
    \end{tabular} 
    \end{center}
    }

\usepackage{hyperref}

%Highlight \hl{}
\usepackage{soul}

\usepackage{enumitem}
\newlist{mylist}{itemize}{1}
\setlist[mylist]{label=\textbullet,leftmargin=1cm,nosep}

\usepackage{multirow}

% Reduce space between figure and caption
%\usepackage{caption}
%\captionsetup[table]{font=small,skip=0pt}     %% Adjust here
%or equivalently 
\usepackage[font=small,skip=4pt]{caption}
%Useful definitions
%\newcommand{\mv}[1]{\textit{m\_#1}}
%\newcommand{\cv}[1]{\textit{c\_#1}}
%\newcommand{\degree}[1]{^{\circ}\mathrm{#1}}
%\newcommand{\comment}[1]{{\footnotesize \quad\texttt{--}\textrm{#1}}}

% Set the header
\ihead[]{\small EECS4313 Assignment-2}


%%%%%%%%%%%%Enter your names here%%%%%%%%
\author{Student Name | Student Number | EECS Account
\and \textbf{Edward Vaisman | 212849857 | eddyv}
\and \textbf{Robin Bandzar | 212200531 | cse23028}
\and \textbf{Kirusanth Thiruchelvam | 212918298 | kirusant}
\and \textbf{Sadman Sakib Hasan | 212497509 | cse23152}
}
%%%%%%%%%%%%%%%%%%%%%%%%%%%%%%%%

\date{\today} % Display a given date or no date

\begin{document}
\title{EECS 4313 Assignment 2 \\Black-box and White-box Testing with JUnit}
\maketitle

\newpage

%%%%%%%%%%%%%%%%%%%%%%%%%%%%%%%
\tableofcontents


\newpage


%%%%Rest of your document goes here%%%%%%%%%%%%%%%%%%%

\section{Black Box Testing}
\begin{itemize}
\item \textbf{Technique}: \emph{Boundary Value Testing}
\item \textbf{Class}: \emph{net.sf.borg.common.SocketClient.java}
\item \textbf{Method}: \emph{sendMsg(String host, int port, String msg)}\\
This method sends a given message to a given host, port and returns the response from the socket.
\begin{itemize}
\item the first argument \emph{host} is the host that the socket client should be connected to.
\item the second argument \emph{port} is the port on the host that the socket client should be connected to
\item the third argument \emph{msg} is the message that should be sent over the host and port given.
\end{itemize}
\begin{description}
\item[UnknownHostException:] If the IP address of the host could not be determined.
\item[IllegalArgumentException:] If the port parameter is outside the specified range of valid port values, which is between 0 and 65535, inclusive.
\item[IOException:] If an I/O error occurs when sending the message.
\end{description}
\item \textbf{Justification}: Boundary value testing is best suited for methods that have inputs that could be seperated into partitions. For this method the port could be partitioned. We have our valid partition which is between 0 and 65535 (inclusive) and our invalid partitions which is any  port\textless  -1 or any port\textgreater 65535.
\newpage
\begin{code}	@Test
	public void test_sendMsg() {
		/** Method used: Boundary Value Testing **/
		String validHost = "localhost";
		String invalidHost = "asdfasdf";
		String msg = "Port 2929";

		int port_norm = 2929; // x_norm
		int port_min = 0; // x_min
		int port_min_plus = 1; // x_min+
		int port_max = 65535; // x_max
		int port_max_minus = 65534; // x_max-

		// robustness
		int port_min_minus = -1; // x_min-
		int port_max_plus = 65536; // x_max_+

		// port_norm
		String response = SocketClient.sendMsg(invalidHost, port_norm, msg);
		response = SocketClient.sendMsg(validHost, port_norm, msg);

		// port_min
		msg = "Port 0";
		response = SocketClient.sendMsg(invalidHost, port_min, msg);
		response = SocketClient.sendMsg(validHost, port_min, msg);

		// port_min+
		msg = "Port 1";
		response = SocketClient.sendMsg(invalidHost, port_min_plus, msg);
		response = SocketClient.sendMsg(validHost, port_min_plus, msg);

		// port_max
		msg = "Port 65535";
		response = SocketClient.sendMsg(invalidHost, port_max, msg);
		response = SocketClient.sendMsg(validHost, port_max, msg);

		// port_max-
		msg = "Port 65534";
		response = SocketClient.sendMsg(invalidHost, port_max_minus, msg);
		response = SocketClient.sendMsg(validHost, port_max_minus, msg);

		// port_min-
		msg = "Port -1";
		response = SocketClient.sendMsg(invalidHost, port_min_minus, msg);
		response = SocketClient.sendMsg(validHost, port_min_minus, msg);

		// port_max+
		msg = "Port 65536";
		response = SocketClient.sendMsg(invalidHost, port_max_plus, msg);
		response = SocketClient.sendMsg(validHost, port_max_plus, msg);

\end{code}
\item Attaching bug reports if bugs are discovered using your testing methods. You should use the
same bug report format as in Assignment 1. Do not file these bug reports to the project’s bug
report system. 
\end{itemize}

\newpage
\section{White Box Testing}

\begin{itemize}
\item The statement coverage measurements for your Assignment 2 test suite.
\item A description of the test cases that you added in this assignment to improve statement
coverage. The marker will not read your code in order to see what you tested. You have to
describe it.
\item The statement coverage measurements for your final submission. Include the screenshots of
the test running results and the screenshots of the coverage measurement. If your coverage is
not 100%, include a discussion on why that is.
\item The Control Flow Graph you created. Indicate the segments clearly (you will probably need
to include the code for this).
\item The path coverage discussion described in section 2 above.
\item Attaching bug reports if bugs are discovered using your testing methods. You should use the
same bug report format as in Assignment 1. Do not file these bug reports to the project’s bug
report system.
\item An appendix with the specification of the methods you are testing (if there are new ones). 
\end{itemize}

\end{document}
